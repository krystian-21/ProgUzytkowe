\documentclass{article}
\usepackage[a4paper,left=3.5cm,right=2.5cm,top=2.5cm,bottom=2.5cm]{geometry}
%%\usepackage[MeX]{polski}
%%\usepackage[cp1250]{inputenc}
\usepackage{polski}
\usepackage[utf8]{inputenc}
\usepackage[tableposition=top]{caption}
\usepackage{algorithmic}
\usepackage{graphicx}
\usepackage{enumerate}
\usepackage{multirow}
\usepackage{amsmath} %pakiet matematyczny
\usepackage{amssymb} %pakiet dodatkowych symboli
\begin{document}
\section{Formuły matematyczne w TeXu}
Przetrenuj używanie w TeXu matematycznych form i symboli z rodziału 1 po czym wykonaj polecenie z rozdziału 2.
\subsection{Zapis Matematyczny}
\subsubsection{Tryb matematyczny}
Tryb matematyczny 'inline' - wzory pisane w lini tekstu wstawiamy przy pomocy \$ wzór \$ (wzór wpisujemy w pojedyncze dolary

Ułamek w tekście $ \frac{1}{x} $ 
Oto równanie $c^{2}=a^{2}+b^{2}$

\noindent 
Ułamek w tekście $ \frac{1}{x}$ 
Oto równanie $c^{2}=a^{2}+b^{2}$


	Tryb matematyczny z zastosowaniem podwójnych dolarów \$\$ wzór \$\$


Ułamek $$ \frac{1}{x} $$ 
Oto równanie $$c^{2}=a^{2}+b^{2}$$



\noindent 
Ułamek $$ \frac{1}{x} $$ 
Oto równanie $$c^{2}=a^{2}+b^{2}$$

Tryb matematyczny z użyciem struktury 'equation'


Ułamek
\begin{equation}
\frac{1}{x}
\label{eq:równanie1}
\end{equation}

Oto równanie 

\begin{equation}
c^{2}=a^{2}+b^{2}
\label{eq:równanie2}
\end{equation}
Można odnieść się do powyższych wzorów wykorzystując polecenie 'eqref{etykieta}'.
Ułamek ma numer (1) a równanie ma numer (2)

	Zad.1.
	
	Przestudiuj trzy powyższe przypadki, zwróć uwagę na różnice w wyświetlaniu i możliwości późniejszego odwołania się do równania. Przepisz je do latex'a i spróbuj odnieść się do równania zdefiniowanych przy pomocy 'equation'.
\end{document}